\documentclass[conference]{sig-alternate-05-2015}
\usepackage{color, xcolor, float, lscape, enumerate, graphicx, url, tabularx, multirow, xspace, hyperref}%times
\usepackage[font=bf, skip=0pt]{caption}
%\usepackage{titlesec}
\usepackage[english]{babel}
\usepackage[autostyle, english = american]{csquotes}

\hypersetup{
  colorlinks,
  citecolor=blue,
  linkcolor=red,
  urlcolor=black}
\newcommand{\note}[1]{{\textcolor{blue}{[#1]}}}
\newcommand{\fixme}[1]{{\textcolor{red}{#1}}}
\newcommand{\citeme}{{\textcolor{red}{[?]}}\xspace}
\newcommand{\todo}[1]{{\textcolor{red}{[#1]}}}
\newcommand{\BfPara}[1]{{\noindent\bf#1.}\xspace}
\newcommand{\vi}{\vspace{5mm}}
\newcommand{\etal}{{\em et al.}\xspace}
\newcommand{\eg}{{\em e.g.,}\xspace}
\newcommand{\ie}{{\em i.e.,}\xspace}
\newcommand{\etc}{{\em etc}\xspace}



\usepackage{fancyvrb}
\usepackage{verbatim}
\begin{document}



\title{Proposal: Cyber-Bullying Detection and Classification System}



\author{Marc Mailloux\\ marcmailloux@knights.ucf.edu \and Rolando Nieves  \\ rolando.j.nieves@knights.ucf.edu \and Maxim Shelopugin\\ maxim.shelopugin@knights.ucf.edu}

\maketitle


\section{Motivation \& Problem Statement}\label{sec:motivation}
Among all the forms of bullying and harassment recognized today, bullying via
platforms hosted on the Internet has proliferated at an alarming rate. The rate
of proliferation has been concerning enough to motivate the United States
Centers for Disease Control and Prevention (CSC) to acquire and report data regarding
\enquote{electronic} bullying (or, as it is more commonly known, \enquote{cyber-bullying}) via its
biennial Youth Risk Behavior Surveillance System (YRBSS) \cite{CBRC_facts2018}.

Collection of data from cyber-bullying victims has helped immensely as it pertains
to awareness and prevention. It should be possible to further improve in both areas if
data collection overcomes the limitation of the \enquote{a posteriori} nature of current
methods (i.e., prevention methods derived from data voluntarily disclosed by
victims after harassment has occurred). An automated system capable of examining
electronic communication traffic, identifying and classifying any traffic that
could be considered as bullying, could lead to solutions that can either filter
out any offensive content automatically, or significantly shorten the response
time to an incident, possibly preventing the more dire secondary effects of
cyber-bullying.


\section{Related Work}\label{sec:related}

There has been much research done around this topic, as evidenced by the body of
work on display on the Cyber-Bullying Research Center research summary page
\cite{CBRC_research2018}. Approaches employed to date include relatively
simple Bayes type classifiers, as well as more complex deep learning systems
(Sweta Agrawal et. al. \cite{agrawal2018deep}). One existing implementation that
serves as an excellent exemplar to this proof-of-concept proposal is the
\textit{Anti Bully} project by Michelle Li \cite{Li2016}. The work in this
proposal differs from \textit{Anti Bully} in two important ways:

\begin{itemize}
  \item The system proposed in this paper will not rely solely on Naive Bayes
  classification algorithms the way \textit{Anti Bully} does.
  \item The proposed system will be able to further refine the binary
  classification done in \textit{Anti Bully} (which only identifies traffic as
  \textbf{bullying} or \textbf{not bullying}) by categorizing bullying traffic
  among one of the classes listed in Section \ref{sec:design}.
\end{itemize}

The \textit{Anti Bully} system code base does include a good data set which can
be used in the proof-of-concept proposed in this paper, albeit with some
modifications (see Section \ref{sec:dataset}).

\section{Cyber-Bullying Detection System}\label{sec:design}
The proposal presented in this paper advocates for the implementation of a
proof-of-concept meeting the desired features as outlined in Section \ref{sec:motivation}.
Leveraging technological advances in Natural
Language Processing, along with algorithms borrowed from the areas of Artificial
Intelligence and Machine Learning, the system will be trained to recognize and
classify cyber-bullying traffic. The classification will go beyond simply
identifying bullying traffic, but will also categorize the bullying into three
(3) distinct classes:
\begin{description}
    \item[Cultural Harassment] Offensive content primarily focusing on the victim's race or religious beliefs.
    \item[Sexual Harassment] Amplification of stereotypical behavior for a given gender, or gender identification.
    \item[Personal Attacks] Ridicule based on a victim's outward-visible attributes, such as appearance, mannerisms, or intelligence.
\end{description}

The proposed system, at its core, will be very similar to classification systems
common in the areas of Artificial Intelligence and Machine Learning. The system
will combine a classifying apparatus based on machine learning algorithms, such as
Artificial Neural Networks (ANN), k-Nearest Neighbor (kNN), or Support Vector
Machines (SVM), with a natural language vector representation mechanism, such as
\textit{Word2Vec}.

\section{Evaluation}\label{sec:evaluation}

The primary focus while evaluating the system's performance will be to maximize
detection of bullying traffic (i.e., maximize true positives while minimizing
false negatives), with the minimization of otherwise innocuous traffic (i.e.,
maximizing true negatives while minimizing false positives).

Focusing on the \textit{Recall} classification metric (i.e., the proportion of true
positives classified properly) first and foremost will provide a clear assessment
regarding the aforementioned behavior we seek from the system. A system that
perfectly identifies all cyber bullying instances while not misclassifying any
innocuous traffic would exhibit a \textit{Recall} score of 1.0. Although such perfect
performance is likely not attainable, tuning of the system will focus on
maximizing the \textit{Recall} score.

Upon quick inspection, no single supervised learning classifier seems like an
out-right favorite to perform optimally in the problem domain addressed by the
proof-of-concept system. Thus, the team will implement various classifiers that
can easily plug into the proof-of-concept system. Only one classifier will be
used at any one time, and a 95\% confidence interval statistical analysis will
determine which one of the classifiers, if any, is significantly better than the
others.

\subsection{Data Set}\label{sec:dataset}
The data set we propose to use in this proof-of-concept, as alluded to in Section
\ref{sec:related}, will be a modified version of the data set provided by the
\textit{Anti Bully} system \cite{Li2016}. The data set as provided is labeled,
but the labels only contain two (2) classes: \textbf{bullying} and
\textbf{not bullying}. In order to meet the goals for the system as detailed in
Section \ref{sec:motivation}, the data set labeling will have to be
enhanced such that the samples labeled as \textbf{bullying} are distributed
among the categories identified in Section \ref{sec:design}.

\section{Expected outcomes and risk management}\label{sec:expectations}

Similar classification exercises, such as that documented by Sweta Agrawal
\cite{agrawal2018deep}, achieved an average Recall score of \( 0.87 \). Thus,
we expect this proof-of-concept to at least meet this performance baseline.

One risk that may lead to a revision of the proof-of-concept design lays with
the data set. As it stands now, we have not been able to ascertain how many of the
samples in the \textit{Anti Bully} data set fit into the categories as listed in
Section \ref{sec:design}. Some of the categories may be either severely
under-represented or not present at all. Should that be the case, the system
will either need to employ classification weighting (to account for data set
class imbalance) or classification re-design (for the case when a category is
not found in the data set).

\section{Plan and Roles of Collaborators}\label{sec:plan_roles}
The proof-of-concept implementation exercise will be divided into five (5)
primary task areas. The order in which the tasks are presented in this paper
does not necessarily represent the chronological order in which the tasks will
be carried out.

\subsection{Data Set Labeling}\label{sec:labeling_task}
As stated in Section \ref{sec:dataset}, the source data set for this
proof-of-concept will need to be refined in order to accommodate the
classification taxonomy as described in Section \ref{sec:design}. Although all
team members will contribute to this task, \textbf{Mr. Marc Mailloux} will serve
as the task's point of contact.

\subsection{Text Representation}\label{sec:tokenization_task}
Transforming the training and test sets from plain english into a form that the
proof-of-concept system can ingest is fundamental. For this task,
\textbf{Mr. Marc Mailloux} will serve as both the primary developer and the
point of contact.

\subsection{Classifiers}\label{sec:classifier_task}
The classifier task, given its wide breadth, will be divided among all team
members. A list of the classifiers that will be used, along with a short
explanation justifying their use, as well as the classifier's implementation
point of contact, follows:
\begin{itemize}
  \item Artificial Neuron Network (ANN) - ANNs possibly represent the most
  flexible of classifiers, both in the form of input they accept as well as the
  output they produce. The point of contact for this classifier will be
  \textbf{Mr. Rolando Nieves}.
  \item Support Vector Machines (SVM) - SVMs have the possibility of requiring
  the least amount of computing power during both training (as compared with
  ANNs) and classification. The point of contact for this classifier will be
  \textbf{Mr. Marc Mailloux}.
  \item Naive Bayes - This was the classifier used in the \textit{Anti-Bully}
  system this proof-of-concept will be based on. It will be interesting to
  assess the impact, if any, of implementing the same classifier on a
  multi-class environment. The point of contact for this classifier will be
  \textbf{Mr. Maxim Shelopugin}.
\end{itemize}

The statistical significance testing, as described in Section
\ref{sec:evaluation}, will be implemented by \textbf{Mr. Rolando Nieves}. Mr.
Nieves will also serve as the point of contact for the classifier task.

\subsection{Presentation}\label{sec:presentation_task}
\textbf{Mr. Maxim Shelopugin} will be responsible for producing any materials
required to present the results of this proof-of-concept to interested parties,
including a summary poster.

\subsection{Final Report}\label{sec:report_task}
Although all team members will contribute content to it,
\textbf{Mr. Rolando Nieves} will be responsible for producing the final report
that will document the results observed while exercising the proof-of-concept
system documented in this proposal.

\bibliographystyle{ieeetr}
\bibliography{proposal}


\end{document}
